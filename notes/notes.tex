\input article
\input macros

\def\baseDir{/afs/cern.ch/work/j/jkaspar/analyses/elastic/6500GeV/beta2500/4rp/}

\def\section#1{%
	\advance\nsec1
	\nsubsec=0 
	\nssubsec=0%
	\penalty-\clubpenalty
	\vskip2\baselineskip
	{%
		\baselineskip15pt
		\vbox{%
			\noindent\SetFontSizesXII\bf\the\nsec. #1%
		}%
	}%
	\penalty\clubpenalty
	\vskip0\baselineskip
	\parindent=0pt
	\everypar={\parindent=\ParIndent \everypar={}}%
}

\def\subsection#1{%
	\advance\nsubsec1
	\nssubsec=0%
	\penalty-\clubpenalty
	\vskip2\baselineskip
	{%
		\vbox{%
			\noindent\bf\the\nsec.\the\nsubsec. #1%
		}%
	}%
	\penalty\clubpenalty
	\vskip0\baselineskip
	\parindent=0pt
	\everypar={\parindent=\ParIndent \everypar={}}%
}

\def\linkColor{\cBlue}

%----------------------------------------------------------------------------------------------------

\centerline{\SetFontSizesXX Elastic analysis, $\sqrt s = 13\un{TeV}$, $\be^* = 2500\un{m}$ }
\vskip2mm
\centerline{\SetFontSizesXX analysis with 4 RPs }
\vskip2mm
\centerline{version: {\it \number\day. \number\month. \number\year}}

%----------------------------------------------------------------------------------------------------
\section{Analysis approach}

\> RPs used (track required): 220-fr unit in both arms



%----------------------------------------------------------------------------------------------------
\section{Datasets}

\> fills: 5313, 5314, 5317, 5321



%----------------------------------------------------------------------------------------------------
\section{Ntuples}

\> current: /eos/totem/data/offline/2016/2500m/version1/



%----------------------------------------------------------------------------------------------------
\section{Beam conditions}

\> naming
\>> ``unresolved'' activity: event where RP has sufficient number of planes on, but no track is reconstructed

\> plots of rates vs.~time
\>> \plot{common/rates_vs_time.pdf}: basic rates
\>> \plot{common/multitrack_rates_vs_time.pdf}: details for the unresolved category



%----------------------------------------------------------------------------------------------------
\section{Reconstruction formulae}

\> method: study various reconstruction formulae with MC

\> \plot{reconstruction_formulae/plot_formulae_graphs.pdf} : study of performance of various reconstruction formulae
\>> formulae
\>>> using different RPs as input
\>>> combining the input in various ways
\>> studying the impact of various effects: beam divergence, vertex, pitch, ...

\> \plot{reconstruction_formulae/plot_formulae_correlation.pdf} : study of left-right correlations of angle reconstruction errors

\> observations for horizontal plane
\>> single arm: best is the ``regression'' formula with vertex-contribution suppression; the leading error comes from detector resolution
\>> double arm: vertex cancels automatically due to the optics symmetry, thus:
\>>> best: LR average of ``hit'' formula
\>>> LR average of ``regression'' formula is worse -- the error induced by detector resolution is amplified by the (unnecessary) vertex-suppression on single-arm level
\>> therefore two different reconstructions used in the analysis: ``regression'' formula for cuts, ``hit'' formula for the rest

\> formulae used for cuts
\>> horizontal plane
\>>> single arm: $\displaystyle \th_x^{*,LR} = \mp { v_x^{210F}\,x^{220F} - v^{220F}\,x^{210F} \over v^{210F}\,L^{220F} - v^{220F}\,L^{210F} }$,
$\displaystyle x^{*,LR} = { x^{210F}\,L_x^{220F} - x^{220F}\,L_x^{210F} \over v^{210F}\,L^{220F} - v^{220F}\,L^{210F} }$
\>>> double arm: average of single arm, $\th_x^* = (\th_x^{*,R} + \th_x^{*,L}) / 2$
\>> vertical plane
\>>> single arm: $\displaystyle \th_y^{*,LR} = \mp {1\over 2} \left( {y^{210F}\over L_y^{210F}} + {y^{220F}\over L_y^{220F}} \right)$
\>>> double arm: average of single arm, $\th_y^* = (\th_y^{*,R} + \th_y^{*,L}) / 2$

\> formulae used for the rest (distributions etc.)
\>> horizontal plane
\>>> single arm: $\displaystyle \th_x^{*,LR} = \mp {1\over 2} \left( {x^{210F}\over L_x^{210F}} + {x^{220F}\over L_x^{220F}} \right)$
\>>> double arm: average of single arm, $\th_x^* = (\th_x^{*,R} + \th_x^{*,L}) / 2$
\>> vertical plane
\>>> single arm: $\displaystyle \th_y^{*,LR} = \mp {1\over 2} \left( {y^{210F}\over L_y^{210F}} + {y^{220F}\over L_y^{220F}} \right)$
\>>> double arm: average of single arm, $\th_y^* = (\th_y^{*,R} + \th_y^{*,L}) / 2$

%----------------------------------------------------------------------------------------------------
\section{Alignment}

\subsection{Determination}

\> method: standard procedure
\>> beam-based alignment: before data-taking
\>> track-based alignment: relative alignment between RP sensors
\>> alignment with elastics: absolute alignment wrt.~LHC beam

\> uncertainty of track-based alignment
\>> horizontal shift: $5\un{\mu m}$
\>> vertical shift: $5\un{\mu m}$
\>> rotation about $z$: $1\un{mrad}$

\> uncertainty of elastic alignment
\>> horizontal shift: $25\un{\mu m}$
\>> vertical shift: $100\un{\mu m}$
\>> rotation about $z$: $2\un{mrad}$

\> induced uncertainties
\>> $\th_x^*$: $0.5\un{\mu rad}$ (220-fr), $0.35\un{\mu rad}$ (210-fr), $0.30\un{\mu rad}$ (single arm), $0.22\un{\mu rad}$ (double arm)
\>> $\th_y^*$: $0.41\un{\mu rad}$ (210-fr), $0.35\un{\mu rad}$ (220-fr), $0.27\un{\mu rad}$ (single arm), $0.19\un{\mu rad}$ (double arm)
\>> rotation about z: $\th_x^* \rightarrow \th_x^* + C \th_y^*$, $\si(C) = 0.0068$ (210-far), $0.0115$ (220-far), $0.0064$ (single arm), $0.0045$ (double arm)

\subsection{Validation: elastic alignment repeated}

\> elastic alignment: method
\>> applied per unit and per time slice
\>> illustrated in \plot{alignment/alignment_method.pdf}
\>>> hits in the top and bottom RP fitted with a ~vertical line (red) $\rightarrow$ horizontal alignment and tilt
\>>> build $y$ histograms from top and bottom RPs, find the centre of symmetry (blue line) $\rightarrow$ vertical alignment
\>> \plot{alignment/alignment_details_DS-fill5313.pdf}: example of technical details of the method

\> concept: current n-tuples contain alignment $\rightarrow$ expect corrections compatible with 0

\> results
\>> \plot{alignment/alignment_DS-fill5313.pdf}
\>> \plot{alignment/alignment_DS-fill5314.pdf}
\>> \plot{alignment/alignment_DS-fill5317.pdf}
\>> \plot{alignment/alignment_DS-fill5321.pdf}

\> observations: results compatible with 0, as expected

\subsection{Validation}

\> \plot{alignment/angular_diff_vs_time.pdf}: differences (left-right, 210m-220m) in reconstructed scattering angles

\> \plot{alignment/mean_th_x_vs_th_y.pdf}: mean of $\th_x^*$ as a function of $\th_y^*$ -- test for residual mis-rotations

\> \plot{alignment/mean_th_x_vs_time.pdf}:  mean of $\th_x^*$ as a function of time -- test for stability of $x$ alignment

\> 2D Gaussian fit of $\th^*_y$ vs $\th^*_x$ distributions combined from the two diagonals (program alignment\_final) gives centre positions
\>> $|\th^*_x| < 0.03\un{\mu rad}$
\>> $|\th^*_y| < 0.08\un{\mu rad}$

\> observations: effects possibly due to misalignment within the tolerances above (induced uncertainties)



%----------------------------------------------------------------------------------------------------
\section{Optics}

\subsection{Validation}

\> \plot{optics/optics_test_summary.pdf}: differences (left-right, 210m-220m) in reconstructed scattering angles plotted as a function of the scattering angle
\>> shifts (fit extrapolated to 0): due to misalignments
\>> tilts (slope): due to optics



%----------------------------------------------------------------------------------------------------
\section{Resolution}

\> method
\>> horizontal plane
\>>> left-right difference of $\th_x^*$ sensitive to beam divergence and detector (RP) resolution
\>>> beam divergence determined from $\si(x^*)$
\>>> detector resolution then deconvolved from $\De^{R-L} \th_x^*$ and $\si(x^*)$ and is checked to be time-independent
\>> vertical plane
\>>> left-right difference of $\th_y^*$ directly probes the vertical beam divergence (the detector component is negligible)

\> \plot{resolutions/resolutions_vs_time.pdf}: summary of resolution components as a function of time



%----------------------------------------------------------------------------------------------------
\section{Cuts/elastic tagging}

\> event selection
\>> track in all 4 RPs of a diagonal
\>> elastic tagging, see below
\>> RP trigger flag: non zero (ev.trigger\_bits \& 7)

\> elastic tagging: general cut structure $| a q_a + b q_b + c| < n_\si  \si$
\>> cut 1: $q_a = \th_x^{*R}$, $q_b = \th_x^{*L}$; \plot{cuts/cut_1.pdf}
\>> cut 2: $q_a = \th_y^{*R}$, $q_b = \th_y^{*L}$; \plot{cuts/cut_2.pdf}
\>> cut 7: $q_a = \th_x^*$, $q_b = x^{*R} - x^{*L}$; \plot{cuts/cut_7.pdf}
\>> \TODO: why not cuts 3+4, cuts 5+6?

\> cuts 3 and 4: no further benefit

\> all cuts applied at $n_\si = 4$ level

\> \plot{cut_parameters_vs_time.pdf} : cut parameters as function of time
\>> \TODO
\iffalse
\>> mean: maximum offset from zero about $0.15\un{\si}$
\>>> negligible impact as cuts applied at $n_\si$
\>> sigma: taking the maximum for cuts
\>>> cut 1: $\si = 14\un{\mu rad}$
\>>> cut 2: $\si = 0.4\un{\mu rad}$
\fi




%----------------------------------------------------------------------------------------------------
\section{Hit distributions}

\> \plot{hit_distributions/hit_distributions.pdf}: hit distribution after elastic event selection



%----------------------------------------------------------------------------------------------------
\section{Background, efficiency and purity of cuts}

\TODO: as in 2RP analysis


\> background studies
\>> \plot{background,cut_efficiency/cut_distributions.pdf}: distributions of cut discriminators under various cut combinations

\>> \plot{background,cut_efficiency/cut_dist_antidgn_cmp.pdf}: distributions of cut discriminators, after all available cuts, comparison between diagonal and anti-diagonal configurations

\>> \plot{background,cut_efficiency/t_distributions_cuts.pdf}: $t$-distributions under various cut combinations

\>> \plot{background,cut_efficiency/t_dist_antidgn_cmp.pdf}: $t$-distributions, after all available cuts, comparison between diagonal (mostly signal) and anti-diagonal (background) configurations


\> study of signal loss due to cuts
\>> \plot{background,cut_efficiency/t_distributions_n_si.pdf}: $t$ distribution at different cut levels



%----------------------------------------------------------------------------------------------------
\section{Acceptance correction}

\> method
\>> known acceptance limitations: sensor edges (low $|\th_y^*|$), LHC apertures (high $|\th_y^*|$)
\>> acceptance region determined empirically by looking at the reconstructed $\th_y^*$ vs.~$\th_x^*$ distributions, separately in the left and right arm
\>> due to the presence of the tilted RPs (210-far), the acceptance limitations due to the sensors edges are both $\th_y^*$ and $\th_x^*$ dependent
\>> assuming that the smearing in $\th_{x, y}^*$ is independent and has the same sigma left and right, then:
$$h_{\rm observed}(\th_x^*, \th_y^*) = A_{\rm sm}(\th_x^*, \th_y^*)\ h_{\rm smeared}(\th_x^*, \th_y^*)$$
where $h$ stands for event distribution, $\th_x^*$ and $\th_y^*$ are the scattering angles reconstructed via left-right average and $A_{\rm sm}$ is the ``smearing'' contribution to the acceptance
\>> $A_{\rm sm}$ is calculated as the probability that any of the two protons leaves the fiducial region due to smearing -- this time it requires two-fold integral
\>> another acceptance factor, $A_{\rm geom}$, is given by the fraction of const-$\th^*$ arc within the fiducial region
\>>> for this, another set of fiducial cuts, ``global'', is used -- in order to avoid to corners with high smearing correction

\> \plot{acceptance_correction/acc_cmp_fill.pdf}: shows the region of acceptance in $\th_x^*$ vs.~$\th_y^*$, per fill

\> \plot{acceptance_correction/acc_cmp_fill_details.pdf}: shows the details of acceptance limitations due to the sensor edges and the corresponding fiducial cuts

\> \plot{acceptance_correction/fiducial_cut_cmp.pdf}: comparison of fiducial cuts in the left and right arm and the global cuts


%\> \plot{acceptance_correction/acc_phi_lab.pdf}



%----------------------------------------------------------------------------------------------------
\section{Efficiency studies, pile-up, ...}

\subsection{DAQ efficiency}

\> method: per time slice calculate from the trigger block:
$$\hbox{DAQ efficiency} = \hbox{recorded events} / \hbox{triggered events}$$

\> \plot{efficiencies/daq_efficiency.pdf}: DAQ efficiency as function of time

\subsection{Trigger efficiency}

\> method:
\>> take zero-bias (BX) data
\>> select elastic events (standard tagging) $\rightarrow$ number of events $N(BX,elastic)$
\>> out of the selected events, check how many have trigger flag $\rightarrow$ number of events $N(BX,elastic,trigger)$
\>> efficiency = $N(BX,elastic,trigger) / N(BX,elastic)$

\> results:
\>> fill 5317, diagonal 45 bot -- 56 top: $N(BX,elastic) = N(BX,elastic,trigger) = 7280$

\subsection{3-out-of-4 efficiencies}

\> method
\>> remove a RP from the elastic selection
\>> redo a simplified analysis without the RP
\>> for tagged events, look how often the studied RP has a track (compatible with the tagged event)

\> study with tracks
\>> \plot{efficiencies/eff3outof4_fits.pdf}: efficiency as function of $\th_y^*$
\>> \plot{efficiencies/eff3outof4_2D.pdf}: efficiency as function of $\th_x^*$ and $\th_y^*$
\>> \plot{efficiencies/eff3outof4_n_si_cmp.pdf}: efficiency as function of $\th_y^*$, for different choices of the selection $n_\si$
\>> \plot{efficiencies/eff3outof4_vs_time.pdf}: efficiency as function of time

\> study with lower-level signals (patterns, hits, ...)
\>> \plot{efficiencies/eff3outof4_details.pdf}: (in)efficiencies as function of $\th_y^*$
\>> \plot{efficiencies/eff3outof4_details_2D.pdf}: (in)efficiencies as function of $\th_x^*$ and $\th_y^*$




\iffalse
\> \plot{efficiencies/eff3outof4.pdf} : single RP inefficiencies WITHOUT the cut in $\th_x^*$

\> \plot{efficiencies/eff3outof4_2D.pdf} : single RP inefficiencies as function of $\th_x^*$ and $\th_y^*$
\>> clearly indicates the inefficiency due to the horizontal RPs
\>> vertical dashed lines show the $\th_x^*$ cut to select only region with reasonable efficiency

\> \plot{efficiencies/eff3outof4_afterCut.pdf} : single RP inefficiencies WITH the cut in $\th_x^*$
\>> red: with the $\th_x^*$ cut, green: without
\>> with cut: plateau flat
\fi

\subsection{Pile-up}

\> method
\>> take BX sample
\>> evaluate probability of signal that could ``hide'' elastic event, typical conditions:
\>>> ``pl\_suff'': sufficient number of planes is on
\>>> ``pat\_suff'': U or V pattern is recognised 

\> \plot{efficiencies/pileup_details.pdf}: contributions from each arm, each RP and their combinations

\> \plot{efficiencies/pileup.pdf}: final inefficiency as a function of time



%----------------------------------------------------------------------------------------------------
\section{Unfolding of resolution effects}

\> not done yet



%----------------------------------------------------------------------------------------------------
\section{Normalisation}

\> not done yet



%----------------------------------------------------------------------------------------------------
\section{Binning}

\> method
\>> at $|t| \approx 10^{-3}\un{GeV^2}$: very fine binning
\>> at $|t| \approx 10^{-1}\un{GeV^2}$: bin size about $1\un{\si}$ of the $t$ smearing
\>> at $|t| \approx 0.25\un{GeV^2}$: bin size for a fixed statistical uncertainty
\>> at $|t| \gs 0.3\un{GeV^2}$: constant bin size to avoid excessively large bins

\> \plot{binning/bin_size_vs_t.pdf} : visualisation of binning determinants vs.~several binnings used in
the analysis

\> currently the ``ob-1-30-0.05'' binning used



%----------------------------------------------------------------------------------------------------
\section{Validation of analysis chain}



%----------------------------------------------------------------------------------------------------
\section{$t$-distributions}

\> \plot{t_distributions/t_dist_fill_cmp.pdf}: comparison of $t$-distributions from different fills and diagonals

\> \plot{t_distributions/t_dist_merged.pdf}: full and low-$t$ plots of the merged $t$-distribution

\> \plot{t_distributions/t_dist_rel.pdf}: merged $t$-distribution in a relative frame


%----------------------------------------------------------------------------------------------------
\section{Systematic uncertainties}

\subsection{Effects one by one}

\> not studied yet

\subsection{Multiple effects}

\> not studied yet



\bye
