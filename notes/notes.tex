\input article
\input macros

\def\baseDir{/afs/cern.ch/work/j/jkaspar/analyses/elastic/6500GeV/beta2500/2rp/}

\NumberSectionstrue

%----------------------------------------------------------------------------------------------------

\centerline{\SetFontSizesXX Elastic analysis, $\sqrt s = 13\un{TeV}$, $\be^* = 2500\un{m}$}
\centerline{\SetFontSizesXX analysis with 2 RPs }
\vskip2mm
\centerline{version: {\it \number\day. \number\month. \number\year}}

%----------------------------------------------------------------------------------------------------
\section{Analysis approach}

\> RPs used (track required): 220-fr unit in both arms

%----------------------------------------------------------------------------------------------------
\section{Datasets}

\> fills: 5313, 5314, 5317, 5321

%----------------------------------------------------------------------------------------------------
\section{Beam conditions}

\> naming
\>> ``unresolved'' activity: event where RP has sufficient number of planes on, but no track is reconstructed

\> plots of rates vs.~time
\>> \plot{common/rates_vs_time.pdf}: basic rates
\>> \plot{common/multitrack_rates_vs_time.pdf}: details for the unresolved category


%----------------------------------------------------------------------------------------------------
\section{Ntuples}

\> currently used: /eos/totem/data/offline/2016/2500m/version1/

%----------------------------------------------------------------------------------------------------
\section{Hit distributions}

\> \plot{hit_distributions/hit_distributions.pdf}: hit distribution after elastic event selection

%----------------------------------------------------------------------------------------------------
\section{Reconstruction formulae}

\> \plot{reconstruction_formulae/plot_formulae_graphs.pdf} : study of performance of various reconstruction formulae
\>> formulae
\>>> using different RPs as input
\>>> combining the input in various ways
\>> studying the impact of various effects: beam divergence, vertex, pitch, ...

\> \plot{reconstruction_formulae/plot_formulae_correlation.pdf} : study of left-right correlations of angle reconstruction errors
\>> significant in $x$ due to neglecting $x^*$

\> formulae used
\>> horizontal plane
\>>> single arm: $x / L_x$
\>>> double arm: average of single arm
\>> vertical plane
\>>> single arm: $y / L_y$
\>>> double arm: average of single arm

%----------------------------------------------------------------------------------------------------
\section{Cuts/elastic tagging}

\> event selection
\>> track in 220-fr units of a diagonal
\>> elastic tagging, see below
\>> RP trigger flag: non zero (ev.trigger\_bits \& 7)

\> elastic tagging: general cut structure $| a q_a + b q_b + c| < n_\si  \si$
\>> cut 1: $q_a = \th_x^{*R}$, $q_b = \th_x^{*L}$; \plot{cuts/cut_1.pdf}
\>> cut 2: $q_a = \th_y^{*R}$, $q_b = \th_y^{*L}$; \plot{cuts/cut_2.pdf}

\> all cuts applied at $n_\si = 4$ level


%----------------------------------------------------------------------------------------------------
\section{Efficiency and purity of cuts, background}

\> background studies
\>> \plot{background,cut_efficiency/cut_distributions.pdf}: distributions of cut discriminators under various cut combinations

\>> \plot{background,cut_efficiency/cut_dist_antidgn_cmp.pdf}: distributions of cut discriminators, after all available cuts, comparison between diagonal and anti-diagonal configurations

\>> \plot{background,cut_efficiency/t_distributions_cuts.pdf}: $t$-distributions under various cut combinations

\>> \plot{background,cut_efficiency/t_dist_antidgn_cmp.pdf}: $t$-distributions, after all available cuts, comparison between diagonal (mostly signal) and anti-diagonal (background) configurations
\>>> background / signal at low $|t|$: $\approx 0.1\un{\%}$
\>>> background / signal at dip: $\approx 10\un{\%}$

\> study of signal loss due to cuts
\>> \plot{background,cut_efficiency/t_distributions_n_si.pdf}: $t$ distribution at different cut levels
\>>> at low $|t|$: inefficiency $\O{0.1\un{\%}}$ and flat in $|t|$

%----------------------------------------------------------------------------------------------------
\section{Efficiency studies, pile-up, ...}

\subsection{DAQ efficiency}

\> method: per time slice calculate from the trigger block:
$$\hbox{DAQ efficiency} = \hbox{recorded events} / \hbox{triggered events}$$

\> \plot{efficiencies/daq_efficiency.pdf}: DAQ efficiency as function of time

\subsection{Trigger efficiency}

\> method:
\>> take zero-bias (BX) data
\>> select elastic events (standard tagging) $\rightarrow$ number of events $N(BX,elastic)$
\>> out of the selected events, check how many have trigger flag $\rightarrow$ number of events $N(BX,elastic,trigger)$
\>> efficiency = $N(BX,elastic,trigger) / N(BX,elastic)$

\> results:
\>> fill 5317, diagonal 45 bot -- 56 top: $N(BX,elastic) = N(BX,elastic,trigger) = 7914$
\>> fill 5317, diagonal 45 top -- 56 bot: $N(BX,elastic) = N(BX,elastic,trigger) = 7194$



\subsection{3-out-of-4 efficiencies}

\> method: derive from results of the 4-RP analysis
\>> take the results for 220-fr pots only

\> uncertainties
\>> inefficiency at $\th_y^* = 0$: $0.3\un{\%}$
\>> slope: $15\un{rad^{-1}}$


\subsection{Pile-up}

\> TODO

\> method
\>> take BX sample
\>> evaluate probability of signal that could ``hide'' elastic event, typical conditions:
\>>> ``pl\_suff'': sufficient number of planes is on
\>>> ``pat\_suff'': U or V pattern is recognised 

\> \plot{efficiencies/pileup_details.pdf}: contributions from each arm, each RP and their combinations

\> \plot{efficiencies/pileup.pdf}: final inefficiency as a function of time

%----------------------------------------------------------------------------------------------------
\section{Alignment}

\subsection{Determination}

\> method: standard procedure
\>> beam-based alignment: before data-taking
\>> track-based alignment: relative alignment between RP sensors
\>> alignment with elastics: absolute alignment wrt.~LHC beam

\> uncertainty of track-based alignment
\>> horizontal shift: $5\un{\mu m}$
\>> vertical shift: $5\un{\mu m}$
\>> rotation about $z$: $1\un{mrad}$

\> uncertainty of elastic alignment
\>> horizontal shift: $25\un{\mu m}$
\>> vertical shift: $100\un{\mu m}$
\>> rotation about $z$: $2\un{mrad}$

\> induced uncertainties
\>> shift in 2-arm $\th_x^*$: $0.35\un{\mu rad}$
\>> shift in 2-arm $\th_y^*$: $0.24\un{\mu rad}$
\>> rotation about z: $\th_x^* \rightarrow \th_x^* + C \th_y^*$, $\si(C) = 0.012$

\subsection{Validation}

\> \plot{alignment/angular_diff_vs_time.pdf}: differences (left-right, 210m-220m) in reconstructed scattering angles

\> \plot{alignment/mean_th_x_vs_th_y.pdf}: mean of $\th_x^*$ as a function of $\th_y^*$ -- test for residual mis-rotations

\> \plot{alignment/mean_th_x_vs_time.pdf}:  mean of $\th_x^*$ as a function of time -- test for stability of $x$ alignment

\> observations (2 arm)
\>> shift in $\th_x^*$: $\De^{R-L} \th_x^*$ up to $1\un{\mu rad}$, the above estimate gives RMS of $0.7\un{\mu rad}$ thus quite compatible; maximum of $\De^{R-L} \th_x^*$ only occurs seldom; for safety take $0.5\un{\mu rad}$ for the two-arm uncertainty
\>> shift in $\th_y^*$: compatible with above estimate $0.24\un{\mu rad}$
\>> rotation about z: $\si(C) \approx 0.005$ significantly better than the above estimate, use this value


%----------------------------------------------------------------------------------------------------
\section{Optics}

\subsection{Validation}

\> \plot{optics/optics_test_summary.pdf}: left-right differences in reconstructed scattering angles plotted as a function of the scattering angle
\>> shifts (fit extrapolated to 0): due to misalignments
\>> tilts (slope): due to optics
\>>> within $0.1\un{\%}$ -- for the moment, use this as uncertainty estimate independently for scaling $\th_x^*$ and $\th_y^*$



%----------------------------------------------------------------------------------------------------
\section{Resolutions}

\> method
\>> horizontal plane
\>>> single arm $\th_x^*$ reconstruction biased by neglecting $x^*$
\>>> this bias in left-right antisymmetric, see \plot{reconstruction_formulae/plot_formulae_correlation.pdf}
\>>> therefore $\De^{R-L} \th_x^*$ is dominated by the vertex term
\>>> and the 2-arm reconstruction (L-R average) is almost free of the bias
\>>> this can not be extracted directly from data, but can be studied with a MC tuned to the 4-RP analysis
\>> vertical plane
\>>> left-right difference of $\th_y^*$ directly probes the vertical beam divergence (the detector component is negligible)

\> \plot{resolutions/resolutions_vs_time.pdf} : time-dependence of observed $\si(\De^{R-L} \th_{x, y}^*)$
\>> solid line: compilation of per-run linear fits (excluding points with too large errors)

\> uncertainties of RMS values
\>> R-L difference of $\th_{x, y}^*$: from the above plot
\>>> x: $0.3\un{\mu rad}$
\>>> y: $0.007\un{\mu rad}$
\>> R-L mean of $\th_{x, y}^*$
\>>> x: from 4-RP analysis plus MC, \TODO
\>>> y: from the above plot, $0.0035\un{\mu rad}$


%----------------------------------------------------------------------------------------------------
\section{Acceptance correction}

\TODO

\iffalse
\> method
\>> known acceptance limitations: sensor edges (low $|\th_y^*|$), LHC apertures (high $|\th_y^*|$)
\>> acceptance region determined empirically by looking at the reconstructed $\th_y^*$ vs.~$\th_x^*$ distributions, separately in the left and right arm
\>> due to the presence of the tilted RPs (210-far), the acceptance limitations due to the sensors edges are both $\th_y^*$ and $\th_x^*$ dependent
\>> assuming that the smearing in $\th_{x, y}^*$ is independent and has the same sigma left and right, then:
$$h_{\rm observed}(\th_x^*, \th_y^*) = A_{\rm sm}(\th_x^*, \th_y^*)\ h_{\rm smeared}(\th_x^*, \th_y^*)$$
where $h$ stands for event distribution, $\th_x^*$ and $\th_y^*$ are the scattering angles reconstructed via left-right average and $A_{\rm sm}$ is the ``smearing'' contribution to the acceptance
\>> $A_{\rm sm}$ is calculated as the probability that any of the two protons leaves the fiducial region due to smearing -- this time it requires two-fold integral
\>> another acceptance factor, $A_{\rm geom}$, is given by the fraction of const-$\th^*$ arc within the fiducial region
\>>> for this, another set of fiducial cuts, ``global'', is used -- in order to avoid to corners with high smearing correction

\> \plot{acceptance_correction/acc_cmp_fill.pdf}: shows the region of acceptance in $\th_x^*$ vs.~$\th_y^*$, per fill

\> \plot{acceptance_correction/acc_cmp_fill_details.pdf}: shows the details of acceptance limitations due to the sensor edges and the corresponding fiducial cuts

\> \plot{acceptance_correction/fiducial_cut_cmp.pdf}: comparison of fiducial cuts in the left and right arm and the global cuts


%\> \plot{acceptance_correction/acc_phi_lab.pdf}
\fi



%----------------------------------------------------------------------------------------------------
\section{Unfolding of resolution effects}

\> method (preliminary)
\>> a first fit of data with model including both Coulomb and hadronic contributions
\>> this fit serves as an input (truth) to simulation (numerical integration) with and without smearing effects
\>> ratio of these two curves gives unfolding correction

\> results (preliminary)
\>> for the moment only at low $|t|$: \plot{unfolding/cmp_mc_num_int.pdf}
\>>> as expected: effect in general small (very small beam divergence) and only present at low $|t|$ (strong non-linearity due to Coulomb)




%----------------------------------------------------------------------------------------------------
\section{Binning}

\> method
\>> at $|t| \approx 10^{-3}\un{GeV^2}$: very fine binning
\>> at $|t| \approx 10^{-1}\un{GeV^2}$: bin size about $n\times \un{\si}$ of the $t$ smearing
\>> at $|t| \approx 0.35\un{GeV^2}$: bin size for a fixed statistical uncertainty
\>> if needed at large $|t|$: constant bin size to avoid excessively large bins

\> \plot{binning/bin_size_vs_t.pdf} : visualisation of binning determinants vs.~several binnings used in
the analysis



%----------------------------------------------------------------------------------------------------
\section{Normalisation}

\> method
\>> normalise $\d\si/\d t$ such that $I + S = 29.7\un{mb}$
\>> $I$ stands for $\d\si/\d t$ fit over $0.01 < |t| < 0.05\un{GeV^2}$, integrated over $0 < |t| < 0.01\un{GeV^2}$
\>> $S$ stands for histogram integral of bins $0.01 < |t| < 0.5\un{GeV^2}$


%----------------------------------------------------------------------------------------------------
\section{$t$-distributions}

\> \plot{t_distributions/t_dist_fill_cmp.pdf}: comparison of $t$-distributions from different fills and diagonals

\> \plot{t_distributions/t_dist_merged.pdf}: full and low-$t$ plots of the merged $t$-distribution

\> \plot{t_distributions/t_dist_rel.pdf}: merged $t$-distribution in a relative frame


%----------------------------------------------------------------------------------------------------
\section{Systematic uncertainties}

\subsection{Effects one by one}

\> not studied yet

\subsection{Multiple effects}

\> not studied yet



\bye
