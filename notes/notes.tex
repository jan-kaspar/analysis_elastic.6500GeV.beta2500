\input article
\input macros

\def\baseDir{/afs/cern.ch/work/j/jkaspar/analyses/elastic/6500GeV/beta2500/4rp/}

\def\section#1{%
	\advance\nsec1
	\nsubsec=0 
	\nssubsec=0%
	\penalty-\clubpenalty
	\vskip2\baselineskip
	{%
		\baselineskip15pt
		\vbox{%
			\noindent\SetFontSizesXII\bf\the\nsec. #1%
		}%
	}%
	\penalty\clubpenalty
	\vskip0\baselineskip
	\parindent=0pt
	\everypar={\parindent=\ParIndent \everypar={}}%
}

\def\subsection#1{%
	\advance\nsubsec1
	\nssubsec=0%
	\penalty-\clubpenalty
	\vskip2\baselineskip
	{%
		\vbox{%
			\noindent\bf\the\nsec.\the\nsubsec. #1%
		}%
	}%
	\penalty\clubpenalty
	\vskip0\baselineskip
	\parindent=0pt
	\everypar={\parindent=\ParIndent \everypar={}}%
}

\def\linkColor{\cBlue}

%----------------------------------------------------------------------------------------------------

\centerline{\SetFontSizesXX Elastic analysis, $\sqrt s = 13\un{TeV}$, $\be^* = 2500\un{m}$ }
\vskip2mm
\centerline{\SetFontSizesXX analysis with 4 RPs }
\vskip2mm
\centerline{version: {\it \number\day. \number\month. \number\year}}

%----------------------------------------------------------------------------------------------------
\section{Analysis approach}

\> RPs used (track required): 220-fr unit in both arms



%----------------------------------------------------------------------------------------------------
\section{Datasets}

\> fills: 5313, 5314, 5317, 5321



%----------------------------------------------------------------------------------------------------
\section{Ntuples}

\> current: /eos/totem/data/offline/2016/2500m/version2/
\>> planes without alignment excluded from reconstruction

\> old: /eos/totem/data/offline/2016/2500m/version1/
\>> has a reconstruction problem in 45-210-fr-bt
\>>> 4 planes often missing in data
\>>> these 4 planes excluded from track-based alignment
\>>> these 4 planes not aligned with the others
\>>> when these 4 planes available in an event $\rightarrow$ reconstruction bias $\approx 50\un{\mu m}$ apparent in cut 6



%----------------------------------------------------------------------------------------------------
\section{Beam conditions}

\> naming
\>> ``unresolved'' activity: event where RP has sufficient number of planes on, but no track is reconstructed

\> plots of rates vs.~time
\>> \plot{common/rates_vs_time.pdf}: basic rates
\>> \plot{common/multitrack_rates_vs_time.pdf}: details for the unresolved category



%----------------------------------------------------------------------------------------------------
\section{Reconstruction formulae}

\> method: study various reconstruction formulae with MC

\> \plot{reconstruction_formulae/plot_formulae_graphs.pdf} : study of performance of various reconstruction formulae
\>> formulae
\>>> using different RPs as input
\>>> combining the input in various ways
\>> studying the impact of various effects: beam divergence, vertex, pitch, ...

\> \plot{reconstruction_formulae/plot_formulae_correlation.pdf} : study of left-right correlations of angle reconstruction errors

\> observations for horizontal plane
\>> single arm: best is the ``regression'' formula with vertex-contribution suppression; the leading error comes from detector resolution
\>> double arm: vertex cancels automatically due to the optics symmetry, thus:
\>>> best: LR average of ``hit'' formula
\>>> LR average of ``regression'' formula is worse -- the error induced by detector resolution is amplified by the (unnecessary) vertex-suppression on single-arm level
\>> therefore two different reconstructions used in the analysis: ``regression'' formula for cuts, ``hit'' formula for the rest

\> formulae used for cuts
\>> horizontal plane
\>>> single arm: $\displaystyle \th_x^{*,LR} = \mp { v_x^{210F}\,x^{220F} - v^{220F}\,x^{210F} \over v^{210F}\,L^{220F} - v^{220F}\,L^{210F} }$,
$\displaystyle x^{*,LR} = { x^{210F}\,L_x^{220F} - x^{220F}\,L_x^{210F} \over v^{210F}\,L^{220F} - v^{220F}\,L^{210F} }$
\>>> double arm: average of single arm, $\th_x^* = (\th_x^{*,R} + \th_x^{*,L}) / 2$
\>> vertical plane
\>>> single arm: $\displaystyle \th_y^{*,LR} = \mp {1\over 2} \left( {y^{210F}\over L_y^{210F}} + {y^{220F}\over L_y^{220F}} \right)$
\>>> double arm: average of single arm, $\th_y^* = (\th_y^{*,R} + \th_y^{*,L}) / 2$

\> formulae used for the rest (distributions etc.)
\>> horizontal plane
\>>> single arm: $\displaystyle \th_x^{*,LR} = \mp {1\over 2} \left( {x^{210F}\over L_x^{210F}} + {x^{220F}\over L_x^{220F}} \right)$
\>>> double arm: average of single arm, $\th_x^* = (\th_x^{*,R} + \th_x^{*,L}) / 2$
\>> vertical plane
\>>> single arm: $\displaystyle \th_y^{*,LR} = \mp {1\over 2} \left( {y^{210F}\over L_y^{210F}} + {y^{220F}\over L_y^{220F}} \right)$
\>>> double arm: average of single arm, $\th_y^* = (\th_y^{*,R} + \th_y^{*,L}) / 2$

%----------------------------------------------------------------------------------------------------
\section{Alignment}

\subsection{Determination}

\> method: standard procedure
\>> beam-based alignment: before data-taking
\>> track-based alignment: relative alignment between RP sensors
\>> alignment with elastics: absolute alignment wrt.~LHC beam

\> uncertainty of track-based alignment
\>> horizontal shift: $5\un{\mu m}$
\>> vertical shift: $5\un{\mu m}$
\>> rotation about $z$: $1\un{mrad}$

\> uncertainty of elastic alignment
\>> horizontal shift: $25\un{\mu m}$
\>> vertical shift: $100\un{\mu m}$
\>> rotation about $z$: $2\un{mrad}$

\> induced uncertainties
\>> $\th_x^*$: $0.5\un{\mu rad}$ (220-fr), $0.35\un{\mu rad}$ (210-fr), $0.30\un{\mu rad}$ (single arm), $0.22\un{\mu rad}$ (double arm)
\>> $\th_y^*$: $0.41\un{\mu rad}$ (210-fr), $0.35\un{\mu rad}$ (220-fr), $0.27\un{\mu rad}$ (single arm), $0.19\un{\mu rad}$ (double arm)
\>> rotation about z: $\th_x^* \rightarrow \th_x^* + C \th_y^*$, $\si(C) = 0.0068$ (210-far), $0.0115$ (220-far), $0.0064$ (single arm), $0.0045$ (double arm)

\subsection{Validation: elastic alignment repeated}

\> elastic alignment: method
\>> applied per unit and per time slice
\>> illustrated in \plot{alignment/alignment_method.pdf}
\>>> hits in the top and bottom RP fitted with a ~vertical line (red) $\rightarrow$ horizontal alignment and tilt
\>>> build $y$ histograms from top and bottom RPs, find the centre of symmetry (blue line) $\rightarrow$ vertical alignment
\>> \plot{alignment/alignment_details_DS-fill5313.pdf}: example of technical details of the method

\> concept: current n-tuples contain alignment $\rightarrow$ expect corrections compatible with 0

\> results
\>> \plot{alignment/alignment_DS-fill5313.pdf}
\>> \plot{alignment/alignment_DS-fill5314.pdf}
\>> \plot{alignment/alignment_DS-fill5317.pdf}
\>> \plot{alignment/alignment_DS-fill5321.pdf}

\> observations: results compatible with 0, as expected

\subsection{Validation}

\> \plot{alignment/angular_diff_vs_time.pdf}: differences (left-right, 210m-220m) in reconstructed scattering angles

\> \plot{alignment/mean_th_x_vs_th_y.pdf}: mean of $\th_x^*$ as a function of $\th_y^*$ -- test for residual mis-rotations

\> \plot{alignment/mean_th_x_vs_time.pdf}:  mean of $\th_x^*$ as a function of time -- test for stability of $x$ alignment

\> 2D Gaussian fit of $\th^*_y$ vs $\th^*_x$ distributions combined from the two diagonals (program alignment\_final) gives centre positions
\>> $|\th^*_x| < 0.03\un{\mu rad}$
\>> $|\th^*_y| < 0.08\un{\mu rad}$

\> observations: effects possibly due to misalignment within the tolerances above (induced uncertainties)



%----------------------------------------------------------------------------------------------------
\section{Optics}

\subsection{Validation}

\> \plot{optics/optics_test_summary.pdf}: differences (left-right, 210m-220m) in reconstructed scattering angles plotted as a function of the scattering angle
\>> shifts (fit extrapolated to 0): due to misalignments
\>> tilts (slope): due to optics



%----------------------------------------------------------------------------------------------------
\section{Resolution}

\subsection{Reconstruction for cuts}

\> uses the reconstruction formula with per-arm horizontal-vertex deconvolution

\> method
\>> horizontal plane
\>>> left-right difference of $\th_x^*$ sensitive to beam divergence and detector (RP) resolution
\>>> beam divergence determined from $\si(x^*)$
\>>> detector resolution then deconvolved from $\De^{R-L} \th_x^*$ and $\si(x^*)$ and is checked to be time-independent
\>> vertical plane
\>>> left-right difference of $\th_y^*$ directly probes the vertical beam divergence (the detector component is negligible)

\> \plot{resolutions/resolutions_for_cuts_vs_time.pdf}: summary of resolution components as a function of time
\>> among others gives: horizontal vertex size, horizontal beam divergence, mean RP resolution

\subsection{Reconstruction for distributions}

\> uses the reconstruction formula without explicit horizontal-vertex deconvolution

\> method
\>> horizontal plane
\>>> single arm $\th_x^*$ reconstruction biased by neglecting $x^*$
\>>> this bias in left-right antisymmetric, see \plot{reconstruction_formulae/plot_formulae_correlation.pdf}
\>>> therefore $\De^{R-L} \th_x^*$ is dominated by the vertex term
\>>> and the 2-arm reconstruction (L-R average) is almost free of the bias
\>>> this can not be extracted directly from data, but can be studied with a MC tuned to data results obtained with the reconstruction formula for cuts
\>> vertical plane
\>>> left-right difference of $\th_y^*$ directly probes the vertical beam divergence (the detector component is negligible)

\> \plot{resolutions/resolutions_for_distributions_vs_time.pdf} : time-dependence of observed $\si(\De^{R-L} \th_{x, y}^*)$
\>> solid line: compilation of per-run linear fits (excluding points with too large errors)

\> $d_{x, y}$ (R-L difference of $\th_{x, y}^*$): used for acceptance correction
\>> time dependent
\>> central values and uncertainties from \plot{resolutions/resolutions_for_distributions_vs_time.pdf}
\>> uncertainties:
\>>> x: $0.3\un{\mu rad}$
\>>> y: $0.007\un{\mu rad}$

\> $m_{x, y}$ (R-L mean of $\th_{x, y}^*$): used for resolution unfolding
\>> for the moment: time independent, single value for all fills
\>> uncertainty covers the full time and fill dependence
\>> x: from 4-RP analysis plus MC -- considered two extreme cases
\>>> case 1: $\si(x^*) = 480\un{\mu m}$, $\si^{\rm bd}(\th^*_x) = 0.27\un{\mu rad}$, $\si^{\rm sen}(x) = 11.2\un{\mu m}$
\>>> case 2: $\si(x^*) = 680\un{\mu m}$, $\si^{\rm bd}(\th^*_x) = 0.39\un{\mu rad}$, $\si^{\rm sen}(x) = 10.5\un{\mu m}$
\>>> \plot{reconstruction_formulae/plot_formulae_graphs_desc_th_x.pdf}: MC shows that the impact of vertex in 2-arm reconstruction can be neglected wrt.~beam divergence and sensor resolution
\>>> case 1: MC gives $\si(m_x) = 0.22\un{\mu rad}$
\>>> case 2: MC gives $\si(m_x) = 0.30\un{\mu rad}$
\>>> mean and difference from the two cases: $\si(m_x) = (0.26 \pm 0.04)\un{\mu rad}$
\>> y: from \plot{resolutions/resolutions_for_distributions_vs_time.pdf}, $\si(m_y) = (0.185 \pm 0.010)\un{\mu rad}$




%----------------------------------------------------------------------------------------------------
\section{Cuts/elastic tagging}

\> event selection
\>> track in all 4 RPs of a diagonal
\>> elastic tagging, see below
\>> RP trigger flag: non zero (ev.trigger\_bits \& 7)

\> elastic tagging: general cut structure $| a q_a + b q_b + c| < n_\si  \si$
\>> cut 1: $q_a = \th_x^{*R}$, $q_b = \th_x^{*L}$; \plot{cuts/cut_1.pdf}
\>> cut 2: $q_a = \th_y^{*R}$, $q_b = \th_y^{*L}$; \plot{cuts/cut_2.pdf}
\>> cut 5: $q_a = y^{R,210,F}$, $q_b = y^{R,220,F} - y^{R,210,F}$; \plot{cuts/cut_5.pdf}
\>> cut 6: $q_a = y^{L,210,F}$, $q_b = y^{L,220,F} - y^{L,210,F}$; \plot{cuts/cut_6.pdf}
\>> cut 7: $q_a = \th_x^*$, $q_b = x^{*R} - x^{*L}$; \plot{cuts/cut_7.pdf}

\> cuts 3 and 4: no further benefit

\> all cuts applied at $n_\si = 4$ level

\> \plot{cuts/cut_parameters_vs_time.pdf} : cut parameters as function of time



%----------------------------------------------------------------------------------------------------
\section{Hit distributions}

\> \plot{hit_distributions/hit_distributions.pdf}: hit distribution after elastic event selection



%----------------------------------------------------------------------------------------------------
\section{Background, efficiency and purity of cuts}

\subsection{Background studies}

\> background: non-elastic events passing the tagging cuts

\> ``anti-diagonal`` data
\>> from configurations: 45 top -- 56 top, 45 bot -- 56 bot
\>> sign of $y$ swapped in sector 45 to process the data with the standard chain
\>> cannot contain any signal
\>> background expected similar as in diagonals

\> methods
\>> method 1: plot distributions of cut discriminators under various cut combinations, see\\ \plot{background,cut_efficiency/cut_distributions.pdf}
\>>> central part (signal): unaffected by cuts
\>>> tails (background): drops with increasing number of cuts
\>>> however: unknown interpolation of background from tails to the signal region
\>> method 2: plot distributions of cut discriminators also for anti-diagonal configurations, see\\ \plot{background,cut_efficiency/cut_dist_antidgn_cmp.pdf}
\>>> good agreement in tails -- confirmation that the background is indeed similar in diagonals and anti-diagonals
\>>> anti-diagonals provide shape of interpolation to the signal region -- flat

\> results
\>> \plot{background,cut_efficiency/cut_dist_antidgn_cmp.pdf}: distributions of cut discriminators, after all available cuts, comparison between diagonal and anti-diagonal configurations
\>>> by comparing the signal (diagonals) and background (anti-diagonals) peak: rough estimate of $B/S \ls 10\cdot10^{-4}$

\>> \plot{background,cut_efficiency/t_dist_antidgn_cmp.pdf}: $t$-distributions, after all available cuts, comparison between diagonal (mostly signal) and anti-diagonal (background) configurations
\>>> background only available at low $|t|$ where $\ls 10\cdot10^{-4}$

\> additional material
\>> \plot{background,cut_efficiency/t_distributions_cuts.pdf}: $t$-distributions under various cut combinations
\>>> gives data reduction by each of the cut

\subsection{Study of signal loss due to cuts}

\> method
\>> plot $t$ distributions at different cut levels ($n_\si$)

\> \plot{background,cut_efficiency/t_distributions_n_si_cmp_fill.pdf}: $t$ distribution at different cut levels, for all fills
\>> qualitatively similar results from all fills

\> \plot{background,cut_efficiency/t_distributions_n_si_cmp_binning.pdf}: $t$ distribution at different cut levels, for all binnings
\>> with fine binnings, bands (barely but) visible
\>>> most likely because of the difference in bin contents is a small integer: 1, 2, 3, ...
\>>> for $|t| \gs 0.4\un{GeV^2}$, the relative differences are larger when smaller bins -- large interplay with statistics

\> results
\>> $|t| \ls 0.2\un{GeV^2}$
\>>> difference between $n_\si = 4$ (standard) and $5$ is about $0.5\un{\%}$ and almost flat
\>> $|t| \gs 0.4\un{GeV^2}$
\>>> difference between $n_\si = 4$ (standard) and $5$ is few percent
\>>> differences covered by statistical uncertainties


%----------------------------------------------------------------------------------------------------
\section{Acceptance correction}

\> method
\>> known acceptance limitations: sensor edges (low $|\th_y^*|$), LHC apertures (high $|\th_y^*|$)
\>> acceptance region determined empirically by looking at the reconstructed $\th_y^*$ vs.~$\th_x^*$ distributions, separately in the left and right arm
\>> due to the presence of the tilted RPs (210-far), the acceptance limitations due to the sensors edges are both $\th_y^*$ and $\th_x^*$ dependent
\>> assuming that the smearing in $\th_{x, y}^*$ is independent and has the same sigma left and right, then:
$$h_{\rm observed}(\th_x^*, \th_y^*) = A_{\rm sm}(\th_x^*, \th_y^*)\ h_{\rm smeared}(\th_x^*, \th_y^*)$$
where $h$ stands for event distribution, $\th_x^*$ and $\th_y^*$ are the scattering angles reconstructed via left-right average and $A_{\rm sm}$ is the ``smearing'' contribution to the acceptance
\>> $A_{\rm sm}$ is calculated as the probability that any of the two protons leaves the fiducial region due to smearing -- this time it requires two-fold integral
\>> another acceptance factor, $A_{\rm geom}$, is given by the fraction of const-$\th^*$ arc within the fiducial region
\>>> for this, another set of fiducial cuts, ``global'', is used -- in order to avoid to corners with high smearing correction

\> \plot{acceptance_correction/acc_cmp_fill.pdf}: shows the region of acceptance in $\th_x^*$ vs.~$\th_y^*$, per fill

\> \plot{acceptance_correction/acc_cmp_fill_details.pdf}: shows the details of acceptance limitations due to the sensor edges and the corresponding fiducial cuts

\> \plot{acceptance_correction/fiducial_cut_cmp.pdf}: comparison of fiducial cuts in the left and right arm and the global cuts


%\> \plot{acceptance_correction/acc_phi_lab.pdf}



%----------------------------------------------------------------------------------------------------
\section{Efficiency studies, pile-up, ...}

\subsection{DAQ efficiency}

\> method: per time slice calculate from the trigger block:
$$\hbox{DAQ efficiency} = \hbox{recorded events} / \hbox{triggered events}$$

\> \plot{efficiencies/daq_efficiency.pdf}: DAQ efficiency as function of time

\subsection{Trigger efficiency}

\> method:
\>> take zero-bias (BX) data
\>> select elastic events (standard tagging) $\rightarrow$ number of events $N(BX,elastic)$
\>> out of the selected events, check how many have trigger flag $\rightarrow$ number of events $N(BX,elastic,trigger)$
\>> efficiency = $N(BX,elastic,trigger) / N(BX,elastic)$

\> results:
\>> fill 5317, diagonal 45 bot -- 56 top: $N(BX,elastic) = N(BX,elastic,trigger) = 7280$

\subsection{3-out-of-4 efficiencies}

\> method
\>> remove a RP from the elastic selection
\>> redo a simplified analysis without the RP
\>> for tagged events, look how often the studied RP has a track (compatible with the tagged event)

\> study with tracks
\>> \plot{efficiencies/eff3outof4_fits.pdf}: efficiency as function of $\th_y^*$
\>> \plot{efficiencies/eff3outof4_2D.pdf}: efficiency as function of $\th_x^*$ and $\th_y^*$
\>> \plot{efficiencies/eff3outof4_n_si_cmp.pdf}: efficiency as function of $\th_y^*$, for different choices of the selection $n_\si$
\>> \plot{efficiencies/eff3outof4_vs_time.pdf}: efficiency as function of time

\> study with lower-level signals (patterns, hits, ...)
\>> \plot{efficiencies/eff3outof4_details.pdf}: (in)efficiencies as function of $\th_y^*$
\>> \plot{efficiencies/eff3outof4_details_2D.pdf}: (in)efficiencies as function of $\th_x^*$ and $\th_y^*$




\iffalse
\> \plot{efficiencies/eff3outof4.pdf} : single RP inefficiencies WITHOUT the cut in $\th_x^*$

\> \plot{efficiencies/eff3outof4_2D.pdf} : single RP inefficiencies as function of $\th_x^*$ and $\th_y^*$
\>> clearly indicates the inefficiency due to the horizontal RPs
\>> vertical dashed lines show the $\th_x^*$ cut to select only region with reasonable efficiency

\> \plot{efficiencies/eff3outof4_afterCut.pdf} : single RP inefficiencies WITH the cut in $\th_x^*$
\>> red: with the $\th_x^*$ cut, green: without
\>> with cut: plateau flat
\fi

\subsection{Pile-up}

\> method
\>> take BX sample
\>> evaluate probability of signal that could ``hide'' elastic event, typical conditions:
\>>> ``pl\_suff'': sufficient number of planes is on
\>>> ``pat\_suff'': U or V pattern is recognised 

\> \plot{efficiencies/pileup_details.pdf}: contributions from each arm, each RP and their combinations

\> \plot{efficiencies/pileup.pdf}: final inefficiency as a function of time



%----------------------------------------------------------------------------------------------------
\section{Unfolding of resolution effects}

\> method: the same as in the 2-RP analysis
\>> \TODO: copy relevant description here



%----------------------------------------------------------------------------------------------------
\section{Normalisation}

\> method
\>> normalise $\d\si/\d t$ such that $I + S = 29.7\un{mb}$
\>> $I$ stands for $\d\si/\d t$ fit over $0.01 < |t| < 0.05\un{GeV^2}$, integrated over $0 < |t| < 0.01\un{GeV^2}$
\>> $S$ stands for histogram integral of bins $0.01 < |t| < 0.5\un{GeV^2}$


%----------------------------------------------------------------------------------------------------
\section{Binning}

\> using the same binning as in the 2-RP analysis



%----------------------------------------------------------------------------------------------------
\section{Validation of analysis chain}

\> yet to be done


%----------------------------------------------------------------------------------------------------
\section{$t$-distributions}

\> \plot{t_distributions/t_dist_fill_cmp.pdf} : comparison of $t$-distributions from different fills and diagonals

\> \plot{t_distributions/t_dist_merged_cmp_dgn.pdf} : full and low-$|t|$ plots of the merged $t$-distribution, per diagonal

\> \plot{t_distributions/t_dist_merged_cmp_binning.pdf} : merged and combined $t$-distribution in several binnings, full and low-$|t|$ view

\> \plot{t_distributions/t_dist_rel.pdf} : merged $t$-distribution in a relative reference frame


%----------------------------------------------------------------------------------------------------
\section{Systematic uncertainties}

\subsection{Effects one by one}

\> not studied yet

\subsection{Multiple effects}

\> not studied yet



\bye
